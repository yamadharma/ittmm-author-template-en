% This is a LaTeX template
% for preparing documents for ITTMM conference
%
\documentclass[60x84/16,8pt]{ittmm}

% Don't remove definition.tex
\input{definition}

\begin{document}

{\selectlanguage{english}

\udc{004.4}

\title{ITTMM Conference Thesis Template}

\author[1,2]{A. B. First}
\author[1]{C. D. Second}

\address[1]{Department of Applied Probability and Informatics\\
Peoples' Friendship University of Russia\\
Miklukho-Maklaya str. 6, Moscow, 117198, Russia}
\address[2]{Laboratory of Information Technologies\\
Joint Institute for Nuclear Research\\
Joliot-Curie 6, Dubna, Moscow region, 141980, Russia}

\begin{abstract}
Place here short abstract in English (between 150 and 250 words).
\end{abstract}

\email{\url{first@rudn.university}, \url{second@rudn.university}}

\keywords{computer science, information technologies, conference proceedings}

\tnanks{}

\alttitle{Шаблон оформления рукописи доклада
  на конференцию <<ITTMM>>}

\altauthor[1,2]{А. Б. Первый}
\altauthor[1]{В. Г. Второй}

\altaddress[1]{Кафедра прикладной информатики и теории вероятностей,\\
  Российский университет дружбы народов,\\
  ул. Миклухо-Маклая, д.6, Москва, Россия, 117198}
\altaddress[2]{Лаборатория информационных технологий,\\
Объединённый институт ядерных исследований,\\
ул. Жолио-Кюри 6, Дубна, Московская область, Россия, 141980}

\begin{altabstract}
Разместите здесь аннотацию на русском языке (150--250 слов).
\end{altabstract}

\altkeywords{компьютерные науки, информационные технологии, проведение конференции}


\maketitle

\section{Introduction}
\label{sec:intro}

The proposed submission should describe original work not submitted or
published elsewhere and should begin with an abstract. The complete
text should be 3 pages long in
total, i.e. including all figures, tables and references. 


\section{Main section} 
\label{sec:base-section}

Additional care must be taken when inserting formulas, figures and tables.

Formula
\begin{equation}
a^n+b^n=c^n
\label{eq:Fermat's_Last_Theorem}
\end{equation}
with reference~\eqref{eq:Fermat's_Last_Theorem}.

It is essential that all illustrations are as clear and as legible as
possible. Vector graphics (pdf and eps)~--- instead of rasterized
images~--- should be used for diagrams and schemas whenever
possible. Please check that the lines in line drawings are not
interrupted and have a constant width. Grids and details within the
figures must be clearly legible and may not be written one on top of
the other. Line drawings are to have a resolution of at least 600 dpi
(preferably 1200 dpi). The lettering in figures should not use font
sizes smaller than 6 pt. Figures are to be
numbered and to have a caption which should always be positioned under
the figures, in contrast to the caption belonging to a table, which
should always appear above the table (see Fig.~\ref{fig:logo}).

\begin{figure}
  \centering
  \includegraphics[width=0.4\linewidth]{embl}
  \caption{Logo}
  \label{fig:logo}
\end{figure}

Ensure that all the tables are cited in the text in the correct
order (see Table~\ref{tab:sampletable}). 

\begin{table}[b]
  \centering
  \caption{Table example}
  \label{tab:sampletable}
  \begin{tabular}{|c|c|c|c|c|}
    \hline
    no & $X$ & $Y$ & $R$ & Color\\
    \hline
    1 &     100  &  170 & 30 & red\\
    2 &     100  &  90  & 60 & red\\
    3 &     230  &  250 & 50 & red\\
    4 &     130  &  240 & 60 & red\\
    5 & 300  &      130 & 30 & red\\
    6 &     200  &  150     & 90 & red\\
    \hline
  \end{tabular}
\end{table}

Text fragments of fewer than four lines should not appear at the tops
or bottoms of pages, following a table or figure. In such cases, it is
better to set the figures right at the top or right at the bottom of
the page. A figure should never be placed in the middle of a
paragraph.

The list of references should be given at the end of the text in a
separate section. Citations inserted in the text should use square
brackets and the ordinal number of the item. Numbers should be grouped
where appropriate.

Citation examples: book~\cite{mathtensor,
  jones-fogelin:tcqd}, the section in the book~\cite{Muller2006},
article~\cite{Arduengo1991, Booth1962},
conference proceedings~\cite{Hope2005}.

\section{Conclusions}

The conclusion should contain a brief summary of the study.

\begin{acknowledgments}
  The work is partially supported by RFBR grant No~16-01-20379.
\end{acknowledgments}

% \begin{thebibliography}{99}

% \bibitem{mathtensor}
% L.~Parker, S.~M. Christensen, MathTensor: a system for doing tensor analysis by
%   computer, Addison-Wesley, 1994.

% \bibitem{jones-fogelin:tcqd}
% W.~T. Jones, R.~J. Fogelin, The Twentieth Century to Quine and Derrida, A
%   History of Western Philosophy, Harcourt Brace College Publishers, 1997.

% \bibitem{Muller2006}
% G.~M. Sheldrick, A Short History of SHELXL, International Union of
%   Crystallography and Oxford University Press, 2006.

% \bibitem{Arduengo1991}
% A.~J. Arduengo, III, R.~L. Harlow, M.~Kline, A stable crystalline carbene,
%   J.~Am. Chem. Soc. 113~(1)  361--363.
% \newblock \href {http://dx.doi.org/10.1021/ja00001a054}
%   {\path{doi:10.1021/ja00001a054}}.

% \bibitem{Booth1962}
% G.~Booth, J.~Chatt, The reactions of carbon monoxide and nitric oxide with
%   tertiary phosphine complexes of iron({II}), cobalt({II}), and nickel({II}),
%   J.~Chem. Soc.  2099--2106. \href {http://dx.doi.org/10.1039/JR9620002099}
%   {\path{doi:10.1039/JR9620002099}}.

% \bibitem{Hope2005}
% E.~Hope, J.~Bennett, A.~Stuart, Fluorous zirconium phosphonates: novel
%   inorganic supports for catalysis, in: Pacifichem (International Chemical
%   Congress of Pacific Basin Societies), no. 961, Pacific Basin Chemical
%   Societies.

% \end{thebibliography}

%% You may use bibtex.
\bibliographystyle{elsarticle-num}
\bibliography{main}

\makealttitle

} %END \selectlanguage

\end{document}
